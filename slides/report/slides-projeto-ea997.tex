\documentclass[aspectratio=169]{beamer}
\usepackage[T1]{fontenc}
\usepackage{multicol}
\usepackage{ragged2e}   %new code
\usepackage[utf8]{inputenc}
\usepackage[brazil]{varioref, babel}
\usepackage[square,sort,comma,super]{natbib}
\usepackage{xmpmulti}
\usepackage{epsfig}
\usepackage{caption}
\usepackage{subcaption}
\usepackage{siunitx}
\usepackage{mathtools}
\usepackage{amssymb}
\usepackage{amsmath}
\usepackage{booktabs}
\usepackage{pbox}
\usepackage{graphicx,url}
\usepackage{etoolbox}
\usepackage{ru,hyperref,url} % 
\graphicspath{{../figures/}}
\usepackage{indentfirst}

\usepackage{tikz}
\usetikzlibrary{shapes,arrows,fit, positioning, arrows.meta}
\usetikzlibrary{backgrounds}
\usepgflibrary{shapes.multipart}

\addtobeamertemplate{block begin}{}{\justifying}
\setbeamertemplate{section in toc}[sections numbered]
\setbeamersize{text margin left = 1em}
\setbeamertemplate{caption}[numbered]

\newcommand{\doingles}[1]{\footnote{Do inglês: \emph{#1}.}}

% The title of the presentation:
%  - first a short version which is visible at the bottom of each slide;
%  - second the full title shown on the title slide;

\title{Detecção de doenças cardiovasculares através de sinais de ECG utilizando ferramentas de inteligência artificial}

% Optional: a subtitle to be dispalyed on the title slide
% \subtitle{Show where you're from}

% The author(s) of the presentation:
%  - again first a short version to be displayed at the bottom;
%  - next the full list of authors, which may include contact information;
\author[João Pedro O. Pagnan]{ \footnotesize
  Aluno: João Pedro de Oliveira Pagnan\\
  Professor: Prof. Dr. José Wilson Magalhães Bassani [CEB/UNICAMP]\\\medskip
  }

% The institute:
%  - to start the name of the university as displayed on the top of each slide
%    this can be adjusted such that you can also create a Dutch version
%  - next the institute information as displayed on the title slide
\institute[]{Universidade Estadual de Campinas - Faculdade de Engenharia Elétrica e Computação\\
  EA997 - Introdução à Engenharia Biomédica \\
  }

% Add a date and possibly the name of the event to the slides
%  - again first a short version to be shown at the bottom of each slide
%  - second the full date and event name for the title slide
\date{\scriptsize \today}

\renewcommand{\indent}{\hspace*{2em}}

\begin{document}

\setbeamertemplate{headline}{}
\setbeamertemplate{footline}{}

\begin{frame}
  \titlepage
\end{frame}

% Section titles are shown in at the top of the slides with the current section 
% highlighted. Note that the number of sections determines the size of the top 
% bar, and hence the university name and logo. If you do not add any sections 
% they will not be visible.
\section{Introdução}
\subsection{Identificação de doenças cardiovasculares com ferramentas de IA}
\begin{frame}
    \frametitle{Introdução}
    \justifying 
    \indent
    
    \indent{O diagnóstico de arritmias cardíacas através de sinais de eletrocardiograma (ECG) é de extrema importância para monitorar a saúde do coração através de um método não invasivo \cite{moody2001impact}. Devido a isso, uma boa etapa de interpretação de sinais computadorizados de ECG é fundamental para que o diagnóstico seja feito de forma precisa e, caso exista, a arritmia cardíaca seja detectada corretamente.}
    
    \indent{Embora que esta análise seja tradicionalmente feita por cardiologistas, trabalhos recentes indicam que ferramentas computacionais de aprendizado de máquina \doingles{Machine Learning} podem obter métricas de \textbf{f-medida} e \textbf{acurácia} melhores que as alcançadas por grande parte dos cardiologistas \cite{hannun2019cardiologist, rajpurkar2017cardiologist}.}
  
    \indent{Desta forma, estas ferramentas podem detectar diversos tipos de arritmias cardíacas a partir de uma única derivação com desempenho comparável ao de cardiologistas e, em contextos clínicos, podem reduzir a chance de diagnósticos incorretos e melhorar a interpretação do sinal de ECG de um especialista humano que já terá uma indicação da provável arritmia que o paciente possui \cite{rajpurkar2017cardiologist}.}    
    
\end{frame}

\begin{frame}
	\frametitle{Objetivos}
	\justifying
	
	\indent{Este projeto visa implementar e comparar quatro tipos de classificadores para identificar arritmias cardíacas através de sinais de ECG: um modelo baseado em máquinas de vetores-suporte \doingles{Support Vector-Machine}, outro baseado nos $k$ vizinhos mais próximos \doingles{K-Nearest Neighbors}, um terceiro baseado em florestas aleatórias \doingles{Random Forest} e, por fim, um baseado no tipo de rede neural LSTM \doingles{Long-Short Term Memory} \cite{geron2019hands}.}
	
	\indent{Neste caso, planeja-se também analisar qual a melhor representação para os sinais de ECG, isto é, se representaremos os sinais no domínio do tempo ou da frequência, bem como se o uso de filtros para remoção de ruído pode aprimorar o desempenho dos classificadores.}
\end{frame}

\begin{frame}
	\frametitle{Metodologia}
	\justifying

	\indent{Será utilizada a linguagem \textbf{Python 3}, mais precisamente, as bibliotecas \textbf{Scikit-Learn} e \textbf{TensorFlow} e a base de dados\doingles{Dataset} \textbf{MIT-BIH} \cite{moody2001impact}, que contém dados de 47 pessoas de 23 a 89 anos, incluindo homens e mulheres.}
	
	\indent{Esta base foi construída entre 1975 e 1979 e contém amostras de cinco padrões diferentes de batimento cardíaco: um padrão com o coração saudável, outro com bloqueio do ramo esquerdo, com bloqueio do ramo direito, um quarto com contração atrial prematura e, por fim, um último com contração ventricular prematura. Estes padrões estão ilustrados na figura \ref{fig:padroes-ECG}}.
	
	\indent{O dado foi digitalizado com uma frequência de $\SI{360}{\hertz}$ e cada sinal dura $\SI{1}{s}$, ou seja, cada sinal é um conjunto de $360$ valores.}
	
\end{frame}

\begin{frame}
	\frametitle{Metodologia}
	\justifying

	\begin{figure}[H]
	\centering
	\includegraphics[scale = 0.3]{}
	\caption{Tipos de sinais de ECG presentes na base de dados selecionada.}
	\label{fig:padroes-ECG}
	\end{figure}
\end{frame}

\appendix

\begin{frame}{Referências}
    \tiny
    \bibliography{bib}
    \bibliographystyle{ieeetr}
\end{frame}

\begin{frame}[plain,c]
    \begin{center}
    \Huge Muito Obrigado!
    \end{center}
\end{frame}


\end{document}
