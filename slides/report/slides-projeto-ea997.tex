\documentclass[aspectratio=169]{beamer}
\usepackage[T1]{fontenc}
\usepackage{multicol}
\usepackage{ragged2e}   %new code
\usepackage[utf8]{inputenc}
\usepackage[brazil]{varioref, babel}
\usepackage[square,sort,comma,super]{natbib}
\usepackage{xmpmulti}
\usepackage{epsfig}
\usepackage{subcaption}
\usepackage{siunitx}
\usepackage{mathtools}
\usepackage{amssymb}
\usepackage{amsmath}
\usepackage{booktabs}
\usepackage{pbox}
\usepackage{graphicx,url}
\usepackage{etoolbox}
\usepackage{ru,hyperref,url} % 
\graphicspath{{../figures/}}

\usepackage{tikz}
\usetikzlibrary{shapes,arrows,fit, positioning, arrows.meta}
\usetikzlibrary{backgrounds}
\usepgflibrary{shapes.multipart}

\addtobeamertemplate{block begin}{}{\justifying}
\setbeamertemplate{section in toc}[sections numbered]

% The title of the presentation:
%  - first a short version which is visible at the bottom of each slide;
%  - second the full title shown on the title slide;

\title{Detecção de doenças cardiovasculares através de sinais de ECG utilizando ferramentas de inteligência artificial}

% Optional: a subtitle to be dispalyed on the title slide
% \subtitle{Show where you're from}

% The author(s) of the presentation:
%  - again first a short version to be displayed at the bottom;
%  - next the full list of authors, which may include contact information;
\author[João Pedro O. Pagnan]{ \footnotesize
  Aluno: João Pedro de Oliveira Pagnan\\
  Professor: Prof. Dr. José Wilson Magalhães Bassani [CEB/UNICAMP]\\\medskip
  }

% The institute:
%  - to start the name of the university as displayed on the top of each slide
%    this can be adjusted such that you can also create a Dutch version
%  - next the institute information as displayed on the title slide
\institute[Universidade Estadual de Campinas ]{
  EA997 - Introdução à Engenharia Biomédica \\
  }

% Add a date and possibly the name of the event to the slides
%  - again first a short version to be shown at the bottom of each slide
%  - second the full date and event name for the title slide
\date{\scriptsize \today}

\begin{document}

\setbeamertemplate{headline}{}
\setbeamertemplate{footline}{}

\begin{frame}
  \titlepage
\end{frame}

% Section titles are shown in at the top of the slides with the current section 
% highlighted. Note that the number of sections determines the size of the top 
% bar, and hence the university name and logo. If you do not add any sections 
% they will not be visible.
\section{Introdução}
\subsection{Identificação de doenças cardiovasculares com ferramentas de IA}
\begin{frame}
    \frametitle{Introdução}
    \justifying O desafio de antecipar padrões de comportamento e construir modelos que sejam apropriados para explicar determinados fenômenos da natureza tem importância para várias áreas da ciência.
\end{frame}

\begin{frame}
    \frametitle{Introdução}
    \justifying Uma classe de sistemas dinâmicos particularmente relevante dentro do contexto de modelagem e predição de séries temporais está ligada à ideia de dinâmica caótica. Apesar de serem determinísticos, esses sistemas são extremamente sensíveis às condições iniciais \cite{fiedler1994caos}.
\end{frame}


\subsection{Previsão de séries temporais com redes neurais artificiais}

\begin{frame}
    \frametitle{Introdução}
    \framesubtitle{Previsão de séries temporais com redes neurais artificiais}
    \justifying Tendo em vista o desempenho de modelos não-lineares para previsão de diversas séries temporais \cite{connor1994recurrent}, optamos por estudar a aplicabilidade de redes neurais artificiais à previsão de séries relacionadas a sistemas com dinâmica caótica. 
    
    Esta pesquisa comparou o desempenho de quatro arquiteturas de redes neurais artificiais:
    
    \begin{itemize}[<+-| alert@+>]
	\item \textit{Multilayer Perceptron} (MLP) \cite{rosenblatt1958perceptron},
	\item \textit{Long Short-Term Memory} (LSTM) \cite{connor1994recurrent},
	\item \textit{Gated Recurrent Unit} (GRU) \cite{cho2014learning},
	\item \textit{Echo State Network} (ESN) \cite{jaeger2007echo}.
    \end{itemize} 
\end{frame}


\appendix

\begin{frame}{Referências}
    \tiny
    \bibliography{bib}
    \bibliographystyle{ieeetr}
\end{frame}

\begin{frame}[plain,c]
    \begin{center}
    \Huge Muito Obrigado!
    \end{center}
\end{frame}


\end{document}
